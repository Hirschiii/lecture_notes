\startenvironment env_math
\definemathmatrix[pmatrix]        % defining matrix with parentheses
	[matrix:parentheses]
	[simplecommand=pmatrix]

\definemathmatrix[bmatrix]        % defining matrix with brackets
	[matrix:brackets]
	[simplecommand=bmatrix]
 
\definemathmatrix[determinant]    % defining determinant with bars
	[matrix:bars]
	[simplecommand=thedeterminant]

\setupmathematics[collapsing=3]

\definereferenceformat[eqref][label=Formel]

\defineMPinstance[LTI]  
                 [
                   format=metafun,
                   extensions=yes,
                   initializations=yes,
                   method=double,
                  ]

\startMPdefinitions{LTI}
def pole =
    image(
        draw (1, 1) -- (-1, -1);
        draw (1, -1) -- (-1, 1);
    )
enddef ;

def zero =
    image(
        unfill fullcircle scaled 2;
        draw   fullcircle scaled 2;
    )
enddef;

def PZplot = applyparameters "PZplot" "do_PZplot" enddef ;

presetparameters "PZplot" [
  xmin = -2.5,  xmax = 2.5,
  ymin = -2.5,  ymax = 2.5,
  dx   = 1,     dy   = 1,
  sx   = 5mm,   sy   = 5mm,
  scale = 0.5,

  grid = false,

  poles = { },
  zeros = { },

  style = "\switchtobodyfont[8pt]",

];

vardef do_PZplot =
  image (
    pushparameters "PZplot";

    newnumeric xmin, xmax, ymin, ymax;
    xmin := getparameter "xmin";
    xmax := getparameter "xmax";
    ymin := getparameter "ymin";
    ymax := getparameter "ymax";

    newnumeric sx, sy;
    sx := getparameter "sx";
    sy := getparameter "sy";

    newnumeric dx, dy;
    dx := getparameter "dx";
    dy := getparameter "dy";

    newpath xaxis, yaxis;

    xaxis := (xmin*sx, 0) -- (xmax*sx, 0);
    yaxis := (0, ymin*sy) -- (0, ymax*sy);

    newpath xtick, ytick;
    xtick := (-0.1sx, 0) -- (0.1sx, 0);
    ytick := (0, -0.1sy) -- (0, 0.1sy);

    newstring style;
    style := getparameter "style";

    newboolean grid;
    grid  := getparameter "grid";

    for x = dx step dx until xmax :
        if grid :
            draw yaxis shifted (x*sx, 0) withcolor 0.5white;
        fi
        draw ytick shifted (x*sx, 0);
        label.bot(style & decimal x, (x*sx, 0));
    endfor

    label.lrt(style & "0", origin);

    for x = -dx step -dx until xmin :
        if grid :
            draw yaxis shifted (x*sx, 0) withcolor 0.5white;
        fi
        draw ytick shifted (x*sx, 0);
        label.bot(style & decimal x, (x*sx, 0));
    endfor

    for y = dy step dy until ymax :
        if grid :
            draw xaxis shifted (0, y*sy) withcolor 0.5white;
        fi
        draw xtick shifted (0, y*sy);
        label.lft(style & decimal y, (0, y*sy));
    endfor

    for y = -dy step -dy until ymin :
        if grid :
            draw xaxis shifted (0, y*sy) withcolor 0.5white;
        fi
        draw xtick shifted (0, y*sy);
        label.lft(style & decimal y, (0, y*sy));
    endfor


    drawarrow xaxis;
    drawarrow yaxis;

    label.rt( style & "$\sigma$",  (xmax*sx, 0));
    label.top(style & "$jω$", (0, ymax*sy));

    newpair p ;

    newnumeric scale;
    scale := getparameter "scale";

    if (getparametercount "zeros") > 0 :
          for i = 1 upto getparametercount "zeros":
            p := getparameter "zeros" i;
            draw (zero xysized(scale*sx, scale*sx))
                 shifted (xpart p*sx, ypart p*sy)
                 withpen pencircle scaled 1bp;
            endfor
    fi

    if (getparametercount "poles") > 0 :
          for i = 1 upto getparametercount "poles":
            p := getparameter "poles" i;
            draw (pole xysized(scale*sx, scale*sx))
                 shifted (xpart p*sx, ypart p*sy)
                 withpen pencircle scaled 1bp;
            endfor
    fi


    popparameters;

  )
enddef;
\stopMPdefinitions

\stopenvironment
