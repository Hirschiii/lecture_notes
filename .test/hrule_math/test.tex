\starttext

\startformula
\startalign
\HL
\NC  a + b \NC = c \NR
\NC  d + e \NC = f \NR
\HL
\NC  g + h \NC = i \NR
\HL
\stopalign
\stopformula

\placetable
[here]
[tbl:tablecommands]
{Commands to be used within a table}
{\starttabulate[|l|p(.6\textwidth)|]
\HL
\NC {\bf Command}
\NC {\bf Meaning}
\NR
\HL
\NC \tex{HL}
\NC Inserts a horizontal line
\NR
\NC \tex{NC}
\NC Begins a new column
\NR
\NC \tex{NR}
\NC Begins a new row
\NR
\NC \tex{VL}
\NC Inserts a vertical line delimiting a column (used in place of \tex{NC})
\NR
\NC \tex{NN}
\NC Begins a column in maths mode (used in place of \tex{NC})
\NR
\NC \tex{TB}
\NC Adds some extra vertical space between two rows
\NR
\NC \tex{NB}
\NC Indicates that the next row starts an indivisible block within which there
cannot be a page break
\NR
\HL
\stoptabulate}

\stoptext
