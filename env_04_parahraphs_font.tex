\startenvironment env_04_paragraphs_font

\setupbodyfont[gentium,11pt]     

                         
\setuptolerance[strict] 
\setupalign[hz,hanging]


\definefontsize[e]        % definir une taille de police e à utiliser avec \tfe

\setupbodyfontenvironment           % et lui affecte un facteur de taille de 3x
  [default]
  [e=3.0]


\definemeasure [interligne-toc]              [3.20ex]
\definemeasure [interligne-titre]            [2.80ex]
\definemeasure [interligne-texte]            [3.60ex]        % 2.8ex par défaut
\definemeasure [interligne-textecompact]     [3.00ex]
\definemeasure [interligne-footnotes]        [0.90ex]
\definemeasure [interligne-footnotesencadre] [1.60ex]
\definemeasure [interligne-tableau]          [1.65ex]

\setupwhitespace [big]                        % espace interparagraphes "grand"

\setupinterlinespace                                       % espace interlignes
  [line=\measure{interligne-texte}]


\setupcolumns [n=2,distance=1cm]      % disposition 2 colonnes espacées d'un cm


\setupnarrower[left=4.0cm,right=4.0cm]      % rétrécissement par défaut 4cm g&d

\startsetups recommandations                % rétrécissement pour reco
  \setupnarrower[left=1.5cm,right=3.0cm]
\stopsetups

\setupitemize                  % liste à puce "compactées" et tirets en couleur
  [packed,
   color=high1color]

\definesymbol               % définition d'un symbole = un petit carré de 0.9ex
  [symitemize1]
  [{\blackrule[height=1.3ex,width=0.9ex,depth=-0.4ex,]}]

\setupitemize       % et son utilisation pour le premier niveau de liste à puce
  [1]
  [symbol=symitemize1]

%------------------------------------------------------------------------------

\setupfootnotes                           % #1 gestion des notes de bas de page
  [textcolor=high1color,                     % repérée par un indice en couleur
   textstyle={\rm},                                        % et en police serif
   location=none]                    % mais notes pas écrites en bas de la page
                                                 % on utilisera \placefootnotes

\setupfootnotedefinition          % #2 - mise en forme des notes de bas de page
  [color=maincolor,   % texte de note de la même couleur que le texte principal
   style=sans,                                        % et en police sans serif
   headcolor=high1color,             % la numérotation est colorée différemment
   headstyle=\bf\ss,                                    % et en gras sans serif
   alternative=serried,              % et le texte est mis à la suite du numéro
   distance=0em,                           % sans aucune distance de séparation
   stopper={.~},                % mais derrière chaque numéro : point et espace
   numbercommand=]           % vide sinon, la numérotation est mise en exposant

%------------------------------------------------------------------------------
                  %  pour souligner du texte \highlight[people]{meilleur ami.e}

\definehighlight [people]       [color=high1color]
\definehighlight [membre]       [color=high1color,style=\tfa\ss\bf]
\definehighlight [organisation] [style=\bf]

%------------------------------------------------------------------------------

\setupurl[style=\ss\bf]           % url en gras et sans serif (sinon monospace)

%------------------------------------------------------------------------------

% Truc de Jedi pour éviter les lignes veuves (lignes seules en bas de page) ou
% orphelines (seules en haut de page)
% https://dave.autonoma.ca/blog/2020/04/28/typesetting-markdown-part-8/

\setpenalties\widowpenalties{1}{10000}
\setpenalties\clubpenalties {1}{10000}













\stopenvironment


