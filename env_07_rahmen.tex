\startenvironment env_07_rahmen

%==============================================================================
                                               % on défini une nouvelle section
                                               
\definehead
  [encadre]                    % s'appelant encadre (au lieu de section par ex)
  [section=section-8,                  % il hérite d'un niveau rarement utilisé
   sectionresetset=myreset,    % son syst de reset de la numérotation (cf +bas)
   number=yes,                                   % on veut afficher les numéros
   incrementnumber=yes,                        % et incrémenter la numérotation
   bodypartlabel=myheadchapter,    % un label (cf. les figures ont "figure xx")
   textcolor=high1color,
   textstyle=\bf,
   numbercolor=maincolor,
   numberstyle=\bf,
   before={%
     \blank[2*big]                           % saut de ligne avant de commencer
     \starttextbackground[theoremFrame]       % on va utiliser un fond cf. +bas
     \setupnarrower                                % on règle un rétrécissement
       [left=\measure{margetitre},
        right=0cm]%
     \startnarrower[left,right]                                 % on le démarre
     \startlocalnotes[noteencadre]  % et démarrage notes de bas de page locales
   },
   aftersection={%
     \blank[1*big]                                              % saut de ligne
     {\setupinterlinespace      % on change l'interligne pour les notes de bdep
        [line=\measure{interligne-footnotesencadre}]
      \placelocalnotes[noteencadre]}          % on place les notes de basdepage
      \stoplocalnotes%                          % et on ferme les notes locales
      \stopnarrower%                                        % le rétrécissement
      \stoptextbackground                                             % le fond
      \blank[2*big]},    
  command=\MyTitreEncadre]     % pour le titre, mise en forme spéciale cf + bas

%==============================================================================
                             % mise en place d'un système de numérotation dédié
                % inspiration : https://tex.stackexchange.com/questions/513413/
                    % context-how-to-have-own-section-be-numbered-independently 
                    
\definestructureresetset       % on declare un nouveau système de remise à zéro
  [myreset]                                   % that does not reset any section
  [0,0,0]             % which level you wan to use for reset part|chap|sect|...
  [0]        % default value when not specified (e.g. subsubsubsubsubsection...

  
\setuplist   % on déclare une liste pour "lister" tous les encadrés dans la toc
  [encadre]
  [numbersegments=8]          % on ne note qu'un seul numéro, pas de hiérarchie
  % (puisqu'on ne met jamais à zéro grâce à \definestructureresetset cf + haut)
                 % celui de la section "encadre", qui a été choisie de niveau 8
                                        % numbersegments=encadre est équivalent
                      % numbersegments=3 would give (part|chapt)|section number
                                % numbersegments=3:4 section.subsection numbers

                                
\setupreferencestructureprefix     % il faut faire de même pour la numérotation
  [encadre]                                                    % des références
  [default]
  [prefixsegments=8]
  
                                
\setuplabeltext        % déf. d'un label utilisé dans la numérotation et la toc
  [en]                                                % version langue anglaise
  [myheadchapter={Encadré }]      % le texte lui même associé à "myheadchapter"


%==============================================================================
                                                                % mise en forme

                                                                
%------------------------------------------------------------------------------

\definetextbackground                                          % textbackground 
  [theoremFrame]
  [mp=textframe,                        % qui exploite le graphique "textframe"
   location=paragraph,                      % appliqué aux paragraphes de texte
   style=\ss,
   before={\testpage[5]\blank},     % si <5 lignes dispo sur page; saut de page
                                                       % et on rajoute un blank
   after={\blank[2*medium]}]           % et après on insère deux espaces medium

%------------------------------------------------------------------------------
                         
\definemeasure [margeencadre] [3mm]

%------------------------------------------------------------------------------

\define[2]\MyTitreEncadre{%
\framed
  [frame=off,
   framecolor=green,
   width=\textwidth,
   offset=-0.25px,
   loffset={\the\dimexpr\measure{margeencadre}-\measure{margeencadre}/2\relax},
   toffset=0mm,
   align=flushleft]{%
   %
   \framed
     [frame=off,
      framecolor=high1color,
      location=empty,
      bottomframe=on,
      rulethickness=\measure{epaisseur1},
      offset=-0.25px,
      width={\the\dimexpr\measure{margetitre}-
             \measure{margeencadre}-\measure{margeencadre}\relax},
      align=middle,
      boffset=1mm]
      {#1}%
   %
   \hskip\measure{margeencadre}
   %
   \framed
     [frame=off,
      framecolor=maincolor,
      location=empty,
      bottomframe=on,
      rulethickness=\measure{epaisseur1},
      offset=-0.25px,
      boffset=1mm,
      width={\the\dimexpr(\textwidth - \measure{margetitre})\relax},
      leftmargin=0mm,
      align=flushleft]
      {\setupinterlinespace[line=\measure{interligne-titre}]#2}
}}


%==============================================================================
                                      % système de notes de bas de page locales

\definenote[noteencadre]

\setupnotation    % hérite de \setupfootnotedefinition avec alternative=serried
  [noteencadre]
  [numberconversion=characters,    % numérotation avec des lettres en minuscule
   color=low1color,                    % la note est écrite en couleur discrete
   style={\tf\ss},                                          % et taille normale
   headcolor=high1color,                         % par contre le numéro ressort
   headstyle={\tfa\ss},                                           % en taille a
   distance={\the\dimexpr\measure{margetitre}/2+\measure{margeencadre}/2-0.5em\relax}, % écarté du texte
   numbercommand=,           % vide sinon, la numérotation est mise en exposant
   after={\blank[2*big]},                              % un bon espace derrière
]

\stopenvironment
