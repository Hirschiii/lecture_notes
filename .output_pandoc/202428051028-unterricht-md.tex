\starttext
\chapter[title={2024-08-05 - Trümmerliteratur},reference={Trümmerliteratur}]

\startsectionlevel[title={Was ist
trümmerliteratur},reference={was-ist-trümmerliteratur}]

\startitemize[packed]
\item
  Nachkriegsliteratur
\item
  \startenumerate[n,packed][start=2,stopper=.]
  \item
    Weltkrieg + Nachkriegsjahre bis ca 1968
  \stopenumerate
\item
  Es wurde eine Kultur \quotation{vorgegeben}
\item
  Das 3. Reich schliesst sich an eine demokratie an
\stopitemize

\startsectionlevel[title={Was wird
ausgedrückt},reference={was-wird-ausgedrückt}]

\startitemize[packed]
\item
  Trauer
\item
  Ideologie der Demokratie im kontrast zu des NS zeit
\stopitemize

\stopsectionlevel

\stopsectionlevel

\startsectionlevel[reference={section}]

\startsectionlevel[title={Vergleich:},reference={vergleich}]

\startblockquote
Meinunge zusammenfassen auf Wahrheit und Schönheit eingehen
\stopblockquote

\startplacetable[location=none]
\startxtable
\startxtablehead[head]
\startxrowgroup[lastrow]
\startxrow
\startxcell[width={0.14\textwidth}] Aspekt \stopxcell
\startxcell[width={0.40\textwidth}] Wolfgang Weyrauch \stopxcell
\startxcell[width={0.46\textwidth}] Heinrich Böll \stopxcell
\stopxrow
\stopxrowgroup
\stopxtablehead
\startxtablebody[body]
\startxrow
\startxcell[width={0.14\textwidth}] Schönheit ohne Wahrheit \stopxcell
\startxcell[width={0.40\textwidth}] Wahrheit ohne Schönheit ist besser
als Schönheit ohne Wahrheit (Z. 30) \stopxcell
\startxcell[width={0.46\textwidth}] Die Zeitgenossen sollen nicht in die
\quotation{Idylle} entführwerden (Z. 43). Tendenziell selbe
aussage \stopxcell
\stopxrow
\startxrow
\startxcell[width={0.14\textwidth}] Ohne Literatur keine
Existens \stopxcell
\startxcell[width={0.40\textwidth}] \quotation{Anfang der Existenz ist,
{[}\ldots{}{]} Anfang der Literatur} (Z. 28) \stopxcell
\startxcell[width={0.46\textwidth}]  \stopxcell
\stopxrow
\startxrowgroup[lastrow]
\startxrow
\startxcell[width={0.14\textwidth}] Aufgabe der Literatur \stopxcell
\startxcell[width={0.40\textwidth}] Die \quotation{legitime Wahrheit}
(Z. 30) und die \quotation{Intention der Wahrheit} (Z. 27)
verbreiten \stopxcell
\startxcell[width={0.46\textwidth}]  \stopxcell
\stopxrow
\stopxrowgroup
\stopxtablebody
\startxtablefoot[foot]
\stopxtablefoot
\stopxtable
\stopplacetable

\startsectionlevel[title={Wolfgang
Weyrauch},reference={wolfgang-weyrauch}]

\startitemize[packed]
\item
  Kahlschlag: Alles wird neu gemacht
\item
  Der Krieg darf nicht geschönt werden. Er muss so grausam dargestellt
  werden, wie er war.
\item
  Für einige wird es schwer die Ideologie zu \quotation{wechseln}
  \startitemize[packed]
  \item
    Die alte ist zu sehr eingeprägt
  \stopitemize
\item
  Es soll licht ins dunkle gebracht werden
\item
  Die \quotation{Verschönigung ist} \quotation{Böse} (Z. 30)
\stopitemize

\stopsectionlevel

\startsectionlevel[title={Heinrich Böll},reference={heinrich-böll}]

\stopsectionlevel

\startsectionlevel[title={Unterschiede},reference={unterschiede}]

\startitemize[packed]
\item
  Böll betrachtet die Menschen eher als Opfer
\item
  Böll: Ist ist ein langer prozess diese Ideologie etc aufzuarbeiten.
  Weyrauch hält einen \quotation{cut} für möglich und denk man muss von
  jetzt auf gleich mit der Vergangennheit abschließen
\stopitemize

\stopsectionlevel

\stopsectionlevel

\startsectionlevel[title={Kernthemen der
Trümmerliteratur},reference={kernthemen-der-trümmerliteratur}]

\startitemize[packed]
\item
  Wahrheiten
\item
  Umgang mit der Vergangenheit
\item
  Art aufklärung
\stopitemize

\stopsectionlevel

\stopsectionlevel

\startsectionlevel[title={Wolfgang
Borchert},reference={wolfgang-borchert}]

\startitemize[packed]
\item
  Trockene darstellung
\stopitemize

Nächster Block lesen wir kurzgeschichte

\startitemize[packed]
\item
  ☐ {\bf {\em Buch mitbringen}}
\stopitemize

\stopsectionlevel

\stoptext
