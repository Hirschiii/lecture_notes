

\setvariables[meta]
  [
  title={Abstand eines Punktes von einer Ebene},
subtitle={Einfuehrung},
author={Niklas von Hirschfeld},
date={2024-08-12},
  ]

\starttext

\startfrontmatter
\component c_titlepage

% Table of Contents
\component c_content

\stopfrontmatter

\startbodymatter
\startsectionlevel[title={Aufgaben},reference={aufgaben}]

\startsectionlevel[title={S. 214},reference={s.-214}]

\startsectionlevel[title={Nr. 1},reference={nr.-1}]

\startblockquote
Bestimmen Sie die Koordinaten des Lotfusspunktes F des Lots durch
$A(3|-1|7)$, $B(6|8|19)$ und $C(-3|-3|-4)$ auf der Ebene
$E: \vec{x} = r \cdot \startpmatrix \NC 1 \NR \NC 0 \NR \NC 0 \NR \stoppmatrix + s \cdot \startpmatrix \NC 0 \NR \NC 4 \NR \NC -3 \NR \stoppmatrix$
\stopblockquote

\startformula 
\vec{n} = \startpmatrix
1 \NR
0 \NR
0 \NR
\stoppmatrix \times
\startpmatrix
0 \NR
4 \NR
-3 \NR
\stoppmatrix =
\startpmatrix
0 \NR
3 \NR
4 \NR
\stoppmatrix
 \stopformula

Den Normalenverktor können wir jetzt in die Gleichung mit einsetzten.
Hier setzen wir die Ebenen mit unserer Gerade (Punkt + t *
Normalenverktor) gleich, um den Schnittpunkt zu berechnen.

\startformula 
r \cdot \startpmatrix
1 \NR
0 \NR
0 \NR
\stoppmatrix + s \cdot
\startpmatrix
0 \NR
4 \NR
-3 \NR
\stoppmatrix = \vec{p} + t \cdot \vec{n}
 \stopformula

Nun müssen nurnoch die Ortsvektoren der Punkte in die gleichung
eingegeben. Mit dem Wert für $t$ können wir den Ortsvektor des Punktes
auf der Ebene berechnen, welcher unserem Ausgangspunkt am nächsten
liegt.

\startformula 
f=\vec{p} + t \cdot \vec{n} \NR
 \stopformula

\startplacetable[location=none]
\startxtable
\startxtablehead[head]
\startxrowgroup[lastrow]
\startxrow
\startxcell Punkt \stopxcell
\startxcell Distanz \stopxcell
\stopxrow
\stopxrowgroup
\stopxtablehead
\startxtablebody[body]
\startxrow
\startxcell A \stopxcell
\startxcell 5 \stopxcell
\stopxrow
\startxrow
\startxcell B \stopxcell
\startxcell 25 \stopxcell
\stopxrow
\startxrowgroup[lastrow]
\startxrow
\startxcell C \stopxcell
\startxcell 5 \stopxcell
\stopxrow
\stopxrowgroup
\stopxtablebody
\startxtablefoot[foot]
\stopxtablefoot
\stopxtable
\stopplacetable

\stopsectionlevel

\startsectionlevel[title={A2},reference={a2}]

\startformula 
E: \vec{x} = 
\startpmatrix
1 \NR
3 \NR
-1 \NR
\stoppmatrix + r \cdot
\startpmatrix
1 \NR
0 \NR
-1 \NR
\stoppmatrix + s \cdot
\startpmatrix
5 \NR
2 \NR
0 \NR
\stoppmatrix 
 \stopformula

\startformula 
E: \vec{y} = 
\startpmatrix
0 \NR
1 \NR
-1 \NR
\stoppmatrix + r \cdot
\startpmatrix
0 \NR
1 \NR
1 \NR
\stoppmatrix + s \cdot
\startpmatrix
1 \NR
0 \NR
2 \NR
\stoppmatrix
 \stopformula

\startplacetable[location=none]
\startxtable
\startxtablehead[head]
\startxrowgroup[lastrow]
\startxrow
\startxcell[width={0.33\textwidth}] Punkt \stopxcell
\startxcell[width={0.33\textwidth}] Distanz zu $E: \vec{x}$ \stopxcell
\startxcell[width={0.33\textwidth}] Distanz zu $E: \vec{y}$ \stopxcell
\stopxrow
\stopxrowgroup
\stopxtablehead
\startxtablebody[body]
\startxrow
\startxcell[width={0.33\textwidth}]
$A(0\  | 2\                      | 1)$ \stopxcell
\startxcell[width={0.33\textwidth}] 2 \stopxcell
\startxcell[width={0.33\textwidth}] $\approx 0.41$ \stopxcell
\stopxrow
\startxrow
\startxcell[width={0.33\textwidth}]
$B(1\  | 3\                      | 5)$ \stopxcell
\startxcell[width={0.33\textwidth}] $\approx 5.13$ \stopxcell
\startxcell[width={0.33\textwidth}] $\approx 0.8085$ \stopxcell
\stopxrow
\startxrowgroup[lastrow]
\startxrow
\startxcell[width={0.33\textwidth}]
$C(-3\ | 1\                      | -1)$ \stopxcell
\startxcell[width={0.33\textwidth}] $\approx 4.64079$ \stopxcell
\startxcell[width={0.33\textwidth}] $\approx 2.45$ \stopxcell
\stopxrow
\stopxrowgroup
\stopxtablebody
\startxtablefoot[foot]
\stopxtablefoot
\stopxtable
\stopplacetable

\stopsectionlevel

\startsectionlevel[title={A4},reference={a4}]

\startformula 
E: \vec{x} = 
\startpmatrix
3 \NR
0 \NR
0 \NR
\stoppmatrix + r \cdot
\startpmatrix
0 \NR
3 \NR
0 \NR
\stoppmatrix + s \cdot
\startpmatrix
0 \NR
0 \NR
4 \NR
\stoppmatrix
 \stopformula

\startformula 
P(3|5|7)
 \stopformula

Distanz $= 4.2$

\stopsectionlevel

\startsectionlevel[title={A6},reference={a6}]

\startplacetable[location=none]
\startxtable
\startxtablehead[head]
\startxrowgroup[lastrow]
\startxrow
\startxcell Koordinatenebene \stopxcell
\startxcell Entfernung des Punktes $P(1\|-2\|-3)$ \stopxcell
\stopxrow
\stopxrowgroup
\stopxtablehead
\startxtablebody[body]
\startxrow
\startxcell x1x2 \stopxcell
\startxcell 3 \stopxcell
\stopxrow
\startxrow
\startxcell x1x3 \stopxcell
\startxcell 2 \stopxcell
\stopxrow
\startxrowgroup[lastrow]
\startxrow
\startxcell x2x3 \stopxcell
\startxcell 1 \stopxcell
\stopxrow
\stopxrowgroup
\stopxtablebody
\startxtablefoot[foot]
\stopxtablefoot
\stopxtable
\stopplacetable

Die drei Werte des Punktes $x_1$, $x_2$ und $x_3$ geben so gesehen die
Entfernung zu der jeweiliegen Koordinatenebene ein. Der Punkt
$P(x|y|10)$ ist immer $10$ entfernt von der $x_1x_2$-Koordinatenebenen.

\stopsectionlevel

\startsectionlevel[title={A13},reference={a13}]

Idee: Von beliebigen Punkt aus in die Richtung des Normalenverktors und
dort Ebene Spannen. Punkt haben wir und auch die Spannung mit der
Koordinatengleichung der gegebenen.

\startformula 
E: 4x_1-7x_2+4x_3=6
 \stopformula

Dies trifft zu für ${x_1=2.5, x_2=1, x_3=1}$. Damit können wir anfangen,
die Normalenform der Gleichung bestimmen:

\startformula 
E: 
\startpmatrix
4 \NR
-7 \NR
4 \NR
\stoppmatrix \cdot (
\startpmatrix
x_1 \NR
x_2 \NR
x_3 \NR
\stoppmatrix - 
\startpmatrix
2.5 \NR
1 \NR
1 \NR
\stoppmatrix 
)
 \stopformula

Damit können wir eine Ebene Aufspannen, welche den selben
Normalenverktor hat und so also parrallel verläuft. Um einen Abstand von
$d$ zu erhalten müssen wir nun nur noch eine Gleichung aufstellen.

\startblockquote
Mit dem C.A.S.:
\stopblockquote

\startformula 
solve(
nrom(
\startpmatrix
2.5 \NR
1 \NR
1 \NR
\stoppmatrix -
(
\startpmatrix
2.5 \NR
1 \NR
1 \NR
\stoppmatrix + t \cdot
\startpmatrix
4 \NR
-7 \NR
4 \NR
\stoppmatrix
)) = d, t)
 \stopformula

Damit ergibt sich $t\approx \pm 0.44$. Und dies können wir in die
Parametergleichung mit eingeben.

\stopsectionlevel

\stopsectionlevel

\stopsectionlevel

\stopbodymatter

\startbackmatter
\stopbackmatter

\stoptext
