\starttext
\chapter[title={2024-08-14 - Abstand zweier windschiefer
Geraden},reference={Abstand zweier windschiefer Geraden}]

Bei zwei windschiefen Geraden wird erst eine Hilfebene hinzugezogen.
Diese muss folgende Bedingungen erfüllen:

\startitemize[packed]
\item
  eine der beiden Geraden muss {\bf in} der Ebene liegen
\item
  die andere muss {\bf parrallel} zu ihr verlaufen
\stopitemize

\startParagraph[reference=Aufstellen der Ebene,title=Aufstellen der Ebene]

Die Ebene $E$ enthält die Gerade $g$ und die andere Gerade verläuft
parrallel. Der {\bf Normalenvektor} der Ebene verläuft dabei {\bf
orthogonal} zu den beiden {\bf Richtungsvektoren} der Geraden.

Danach einfach

\stopParagraph

\stoptext
