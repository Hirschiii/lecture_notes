
\setvariables
  [metadata]
  [
  title={Vigenère-Verfahren},
author={Niklas von Hirschfeld},
institute={Gymnasium Lüneburger Heide},
  ]

\starttext

\startcolumnset[twocolumn]

\startblock[title=Einleitung]

Das {\bf Vigenère-Verfahren} ist ein {\em symmetrisches}\footnote{Es
  wird sowohl für die Ver- wie auch Entschlüsselung derselbe Schlüssel
  verwendet.} Verschlüsselungsverfahren, das im 16. Jahrhundert von
Blaise de Vigenère entwickelt wurde. Es gilt als eine Weiterentwicklung
der einfachen Caesar-Verschlüsselung und wurde über Jahrhunderte als
eine der sichersten Methoden zur Verschlüsselung von Texten angesehen.
Anders als beim Caesar-Code wird hier ein {\bf Schlüsselwort} verwendet,
um den Text zu verschlüsseln.

\stopblock

\startblock[title=Funktionsweise]

Das Vigenère-Verfahren nutzt eine Reihe von Caesar-Verschiebungen, die
durch das Schlüsselwort bestimmt werden. Jeder Buchstabe des Klartextes
wird mit einem Buchstaben aus dem Schlüsselwort kombiniert, um einen
verschlüsselten Buchstaben zu erzeugen.

{\bf Beispiel}:

\startitemize[packed]
\item
  Klartext: ATTACKE
\item
  Schlüssel: LEMONLE
\item
  Verschlüsselt: LXFOPVE
\stopitemize

Hier wird jeder Buchstabe des Klartextes entsprechend dem Buchstaben des
Schlüssels verschoben.

\stopblock

\startblock[title=Vorteile]

Ein großer Vorteil des Vigenère-Verfahrens gegenüber der einfachen
Caesar-Verschlüsselung ist die erhöhte {\bf Sicherheit}. Durch die
Verwendung eines mehrstelligen Schlüssels wird eine Häufigkeitsanalyse
deutlich erschwert, da gleiche Buchstaben im Klartext unterschiedlich
verschlüsselt werden können. Dies führte dazu, dass das Verfahren lange
Zeit als \quotation{unbrechbar} galt.

\column

Ein weiterer Vorteil ist die {\bf Flexibilität}: Das Verfahren kann
leicht angepasst werden, indem man das Schlüsselwort ändert, was eine
Vielzahl von möglichen Verschlüsselungen ermöglicht.

\stopblock

\startblock[title=Sicherheit]

Das Vigenère-Verfahren galt bis ins 19. Jahrhundert als sicher, bis
Charles Babbage und später Friedrich Kasiski Methoden entwickelten, um
die Verschlüsselung zu brechen. Das Verfahren ist besonders anfällig für
die {\bf Kasiski-Untersuchung} und die {\bf Häufigkeitsanalyse} über die
wiederkehrenden Buchstabenmuster im Schlüssel.

In der modernen Kryptographie wird das Vigenère-Verfahren als {\bf
unsicher} eingestuft, da es durch fortgeschrittene Methoden leicht
gebrochen werden kann. Dennoch ist es historisch wichtig, da es den Weg
für komplexere Verschlüsselungstechniken ebnete.

Die {\bf Häufigkeitsanalyse} funktioniert hier ähnlich wie bei dem
Caeser-Verfahren. Es wird auch hier der am häufigsten auftauchende
Buchstabe mit dem in der Sprache generell verglichen. Allerdings muss
man hier aufpassen, da die gleichen Buchstaben mit verschiedenen
Schlüsseln verschlüsselt sein könne. Nicht jedes {\em e} ist mit dem
selben Schlüssel verschlüsselt. Um die Häufigkeitsanalyse trozdem
effektiv anwenden zu könne, ist es von vorteil, wenn die {\bf
Schlüssellänge} bekannt ist. Wenn diese zum Beispiel $7$ beträgt, wissen
wir, dass jedes $7$. Wort den selben Schlüssel besitzt. Damit können wir
die Häufigkeitsanalyse auf die gruppierten Buchstaben anwenden.

Wenn die Schlüssellänge nicht bekannt ist, kann die {\bf
Kasiski-Untersuchung} angewendet werden. Diese hat das Zeil, die länge
des Schlüssels herrauszufinden. Dabei werden wiederholte Sequenzen von
Buchstaben in dem codierten Text ermittelt und der Abstand dieser
analysiert. So kann man annäherungsweise an vielfaches des Schlüssels
herausfinden.

\stopblock

\stopcolumnset
\stoptext
