\starttext
\chapter[title={2024-08-12 - Wiederholung und
Einleitung},reference={Wiederholung und Einleitung}]

\startsectionlevel[title={Wiederholung},reference={wiederholung}]

\startitemize[packed]
\item
  Was ist ein koordinaten system
\item
  Was sind Vektoren
  \startitemize[packed]
  \item
    Lagen
  \item
    Gleichungen
  \item
    Kolineare Koordinaten
  \stopitemize
\item
  Was sind Geraden
  \startitemize[packed]
  \item
    Lagen
  \item
    Gleichungen
  \stopitemize
\item
  Was sind Ebenen
  \startitemize[packed]
  \item
    Lagen
  \item
    Gleichungen
  \stopitemize
\stopitemize

\startsectionlevel[title={Vektorprodukt},reference={vektorprodukt}]

\startsectionlevel[title={Kreuzprodukt},reference={kreuzprodukt}]

\startformula 
\begin{aligned}
\vec{u} \times \vec{v} = \begin{pmatrix}
u_2 \cdot v_3 - u_3 \cdot v_2\\
u_3 \cdot v_1 - u_1 \cdot v_3\\
u_1 \cdot v_2 - u_2 \cdot 1
\end{pmatrix}
\end{aligned}
 \stopformula

\type"{image} ./media/kreuzprodukt_hilfe.png :alt: fishy :class: bg-primary :width: 200px :align: center"

Der Normalvektor steht Orthogonal zu einem Vektor, einger Gerade oder
einer Ebene. Für eine Ebene kann er mit dem Kreuzprodukt der
Spannvektroen dieser berechnet werden.

Den Normalvektor Berechnen: $\vec{n} = \vec{AB} \times \vec{AC}$

Normalengleichung:

\startformula 
E: \quad (\vec{x}-\vec{p})\cdot \vec{n}=0
 \stopformula

\startitemize[packed]
\item
  $\vec{p}$ Stützvektor
\stopitemize

\stopsectionlevel

\startsectionlevel[title={Skalar},reference={skalar}]

\stopsectionlevel

\stopsectionlevel

\startsectionlevel[title={Ebenen},reference={ebenen}]

\startformula 
E: \quad \frac{x_1}{u}+\frac{x_2}{v}+\frac{x_3}{w}=1
 \stopformula

\stopsectionlevel

\stopsectionlevel

\stoptext
