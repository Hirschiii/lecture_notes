\starttext
\chapter[title={2024-06-06 - Interferenz Gitter
Versuch},reference={Interferenz Gitter Versuch}]

\startsectionlevel[title={Beobachtung},reference={beobachtung}]

Abstand zum Schirm: 27cm Abstand der Maxima: 12cm

\stopsectionlevel

\startsectionlevel[title={Auswertung},reference={auswertung}]

\stopsectionlevel

\startsectionlevel[title={Aufgaben},reference={aufgaben}]

\startsectionlevel[title={1.},reference={section}]

\startblockquote
Algemein sind folgende Formeln bekannt:
\stopblockquote

\startformula 
\begin{aligned}
\sin{\alpha} = \frac{\lambda}{g} \quad \text{und} \quad \tan{\alpha}=\frac{a}{l}
\end{aligned}
 \stopformula

Wobei $\lambda$ die Wellenlaenge ist.

\startblockquote
Gitter: 500 Spalten pro Millimeter
\stopblockquote

\startformula 
g=\frac{1\cdot 10^{-3}m}{500}=2\cdot 10^{-6}m
 \stopformula

\startitemize[packed]
\item
  $2a_1=0,12m; \quad a_1 = 0,06m; \quad l=27cm=0,27m$
\stopitemize

\startformula 
\begin{aligned}
\lambda&=g \cdot &&\sin{(\tan^{-1}{(\frac{a}{l})})}\\
&= (2\cdot 10^{-6}) \cdot &&\sin{(\tan^{-1}{(\frac{0,12}{0,27})})}\\
&=434\cdot 10^{-9}m
\end{aligned}
 \stopformula

\stopsectionlevel

\stopsectionlevel

\startsectionlevel[title={Versuch
Wiederholung},reference={versuch-wiederholung}]

\startformula 
\begin{aligned}
2a_2 = 0.127m; \quad a_2 = 0.635m; \quad l = 0.38m \\
\end{aligned}
 \stopformula

Berechnung der Wellenlaenge $\lambda$:

\startformula 
\begin{aligned}
\lambda&=g \cdot &&\sin{(\tan^{-1}{(\frac{a}{l})})}\\
&= (2\cdot 10^{-6}) \cdot &&\sin{(\tan^{-1}{(\frac{0,07}{0,38})})}\\
&= 6,34\cdot 10^{-7}m=634nm
\end{aligned}
 \stopformula

\stopsectionlevel

\startsectionlevel[title={Worauf muss man
achten:},reference={worauf-muss-man-achten}]

\startblockquote
Wir sollen naechstes Jahr den Versuch den anderen erklaeren
\stopblockquote

\stopsectionlevel

\startsectionlevel[title={Links},reference={links}]

\startsectionlevel[title={a},reference={a}]

$2a$ ist zwischen den Maxima der Ordnung $n$. Also von einem Maxima bis
zur mitte ist nur $a$

\stopsectionlevel

\stopsectionlevel

\startsectionlevel[title={Zweite Runde},reference={zweite-runde}]

\startitemize[packed]
\item
  2024-06-18
\stopitemize

\startsectionlevel[title={Messung der verschiedenen Wellen /
LED's},reference={messung-der-verschiedenen-wellen-leds}]

\startplacetable[location=none]
\startxtable
\startxtablehead[head]
\startxrowgroup[lastrow]
\startxrow
\startxcell[width={0.25\textwidth}] LED \stopxcell
\startxcell[width={0.25\textwidth}] Wellenlaenge in nm \stopxcell
\startxcell[width={0.25\textwidth}] Abstand 1. Ordnung in
cm\footnote{Abstand 1. Ordnung zur 1. Ordnung} \stopxcell
\startxcell[width={0.25\textwidth}] A. 2. Ordnung \stopxcell
\stopxrow
\stopxrowgroup
\stopxtablehead
\startxtablebody[body]
\startxrow
\startxcell[width={0.25\textwidth}] Rot \stopxcell
\startxcell[width={0.25\textwidth}] 632 \stopxcell
\startxcell[width={0.25\textwidth}] 10,3 \stopxcell
\startxcell[width={0.25\textwidth}] - \stopxcell
\stopxrow
\startxrow
\startxcell[width={0.25\textwidth}] Grün \stopxcell
\startxcell[width={0.25\textwidth}] 514 \stopxcell
\startxcell[width={0.25\textwidth}] 8,5 \stopxcell
\startxcell[width={0.25\textwidth}] 18,8 \stopxcell
\stopxrow
\startxrowgroup[lastrow]
\startxrow
\startxcell[width={0.25\textwidth}] Blau \stopxcell
\startxcell[width={0.25\textwidth}] 463 \stopxcell
\startxcell[width={0.25\textwidth}] 7,5 \stopxcell
\startxcell[width={0.25\textwidth}] 15,7 \stopxcell
\stopxrow
\stopxrowgroup
\stopxtablebody
\startxtablefoot[foot]
\stopxtablefoot
\stopxtable
\stopplacetable

\startformula 
g=\frac{1\cdot 10^{-3}m}{500}=2\cdot 10^{-6}m
 \stopformula

\startsectionlevel[title={Rot},reference={rot}]

\startsectionlevel[title={1. Ordnung},reference={ordnung}]

\startformula 
\begin{aligned}
2a = 0.103m; \quad a = 0.0515m; \quad l = 0.15m \\
\end{aligned}
 \stopformula

Berechnung der Wellenlaenge $\lambda$:

\startformula 
\begin{aligned}
\lambda&=\frac{g}{n} \cdot \sin{(\tan^{-1}{(\frac{a_n}{l})})}\\
&= (2\cdot 10^{-6}) \cdot \sin{(\tan^{-1}{(\frac{0,0515}{0,15})})}\\
&= 6,49\cdot 10^{-7}m
\end{aligned}
 \stopformula

\stopsectionlevel

\stopsectionlevel

\stopsectionlevel

\stopsectionlevel

\startsectionlevel[title={Bedeutung der einzelnen
Bestandteile},reference={bedeutung-der-einzelnen-bestandteile}]

\stopsectionlevel

\stoptext
