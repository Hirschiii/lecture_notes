\starttext
\chapter[title={how_to/unterricht/01_markdown_syntax.md},reference={how_to/unterricht/01_markdown_syntax.md}]

\startsectionlevel[title={Der Komplette mögliche Markdown
syntax},reference={der-komplette-mögliche-markdown-syntax}]

\startsectionlevel[title={This is an H2},reference={this-is-an-h2}]

\startsectionlevel[title={This is an H3},reference={this-is-an-h3}]

\startsectionlevel[title={This is an H4},reference={this-is-an-h4}]

\startParagraph[reference=This is an H5,title=This is an H5]

This is a text paragraph containing an ellipsis \ldots{} and followed by
a thematic break.

\thinrule

This is inline \type{code}. This is a
\goto{link}[url(http://google.cz)]. {\em This is an {\em emphasized}
span of text}. {\bf This is a {\bf strongly emphasized} span of text}.

\startplacefigure[title={example image}]
{\externalfigure[example-image.png]}
\stopplacefigure

/scientists.csv (The great minds of the 19th century rendered via a
content block)

This is a fenced code block:

This is a table:

\startplacetable[title={Demonstration of pipe table syntax.}]
\startxtable
\startxtablehead[head]
\startxrowgroup[lastrow]
\startxrow
\startxcell[align=left] Right \stopxcell
\startxcell[align=right] Left \stopxcell
\startxcell Default \stopxcell
\startxcell[align=middle] Center \stopxcell
\stopxrow
\stopxrowgroup
\stopxtablehead
\startxtablebody[body]
\startxrow
\startxcell[align=left] 12 \stopxcell
\startxcell[align=right] 12 \stopxcell
\startxcell 12 \stopxcell
\startxcell[align=middle] 12 \stopxcell
\stopxrow
\startxrow
\startxcell[align=left] 123 \stopxcell
\startxcell[align=right] 123 \stopxcell
\startxcell 123 \stopxcell
\startxcell[align=middle] 123 \stopxcell
\stopxrow
\startxrowgroup[lastrow]
\startxrow
\startxcell[align=left] 1 \stopxcell
\startxcell[align=right] 1 \stopxcell
\startxcell 1 \stopxcell
\startxcell[align=middle] 1 \stopxcell
\stopxrow
\stopxrowgroup
\stopxtablebody
\startxtablefoot[foot]
\stopxtablefoot
\stopxtable
\stopplacetable

This is a bullet list:

\startitemize
\item
  The first item of a bullet list,
\item
  the second item of a bullet list,
\item
  the third item of a bullet list.
\stopitemize

This is a definition list:

This is a \high{superscript} and a \low{subscript}.

This is a block quote:

\startblockquote
This is the first level of quoting.

\startblockquote
This is nested blockquote.
\stopblockquote

Back to the first level.
\stopblockquote

Here is a note reference\footnote{Here is the note.} and
another.\startbuffer Here's one with multiple blocks.

  Subsequent paragraphs are indented to show that they belong to the
  previous note.

\starttyping
Some code
\stoptyping

  The whole paragraph can be indented, or just the first line. In this
  way, multi-paragraph notes work like multi-paragraph list items.\stopbuffer\footnote{\getbuffer}
Here is an inline note.\footnote{Inlines notes are easier to write,
  since you don't have to pick an identifier and move down to type the
  note.}

This is raw \TeX code:

\startlines
this is a line block that
spans multiple
even discontinuous
lines
\stoplines

This is inline and display TeX math created using dollars signs:

$E=mc^2$

\startformula E=mc^2 \stopformula

\stopParagraph

\stopsectionlevel

\stopsectionlevel

\stopsectionlevel

\stopsectionlevel

\stoptext
