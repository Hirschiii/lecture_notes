\starttext
\chapter[title={2024-06-03 - Dreidimensionale
Bildbearbeitung},reference={Dreidimensionale Bildbearbeitung}]

\startsectionlevel[title={Ressourcen},reference={ressourcen}]

\startitemize[packed]
\item
  S. 201, 202
\stopitemize

\stopsectionlevel

\startsectionlevel[title={Stichpunkte},reference={stichpunkte}]

\startitemize
\item
  Konstrukteur*innen
\item
  Desinger*innen
\item
  Spieleentwickelnde
\item
  bearbeitung räumlicher Darstellung
\item
  auf Computerbildschirm (2D Oberfläche)
\item
  3D auf 2D projektion
\item
  CAD-Programme
\item
  analytische Geometrie liefert komplette funktionalität für die
  entwicklung von CAD-Programmen
\stopitemize

\stopsectionlevel

\startsectionlevel[title={Mutliplikation mit einer
Matrix},reference={mutliplikation-mit-einer-matrix}]

\startformula 
A\cdot \vec{p}=
\startpmatrix
a_{11} \NR a_{12} \NR a_{13}\NC
a_{21} \NR a_{22} \NR a_{23}\NC
a_{31} \NR a_{32} \NR a_{33}
\stoppmatrix \cdot
\startpmatrix
p_1\NC
p_2\NC
p_3\NC
\stoppmatrix =
\startpmatrix
a_{11} \cdot p_{1} \NR+\NR a_{12} \cdot p_{2} \NR+\NR a_{13} \cdot p_{3}\NC
a_{21} \cdot p_{1} \NR+\NR a_{22} \cdot p_{2} \NR+\NR a_{23} \cdot p_{3}\NC
a_{31} \cdot p_{1} \NR+\NR a_{32} \cdot p_{2} \NR+\NR a_{33} \cdot p_{3}
\stoppmatrix =
\startpmatrix
\dot{p_1}\NC
\dot{p_2}\NC
\dot{p_3}\NC
\stoppmatrix
 \stopformula

\stopsectionlevel

\startsectionlevel[title={Projektion auf
Ebenen},reference={projektion-auf-ebenen}]

\startsectionlevel[title={Identische
Abbildung},reference={identische-abbildung}]

\startformula 
\begin{aligned}
A=\begin{pmatrix}
1 & 0 & 0\\
0 & 1 & 0\\
0 & 0 & 1
\end{pmatrix}
\end{aligned}
 \stopformula

\stopsectionlevel

\startsectionlevel[title={Projetkion auf eine
Koordinatenebene},reference={projetkion-auf-eine-koordinatenebene}]

\startformula 
\begin{aligned}
A=\begin{pmatrix}
0 & 0 & 0\\
a & 1 & 0\\
b & 0 & 1\\
\end{pmatrix}
\end{aligned}
 \stopformula

\stopsectionlevel

\startsectionlevel[title={Zentrische Strckung am
Ursprung},reference={zentrische-strckung-am-ursprung}]

\startformula 
\begin{aligned}
\vec{p}=\begin{pmatrix}
z & 0 & 0\\
0 & z & 0\\
0 & 0 & z
\end{pmatrix}
\end{aligned}
 \stopformula

\stopsectionlevel

\startsectionlevel[title={Orthogonale Spigelung an der
x_1-x_3-Ebene},reference={orthogonale-spigelung-an-der-x_1-x_3-ebene}]

\startblockquote
Die $x_1$- und $x_3$-Koordinaten bleiben gleich und die $x_2$-Koordinate
ändert ihr vorzeichen.
\stopblockquote

\startformula 
\begin{aligned}
\vec{p}\cdot \begin{pmatrix}
1 & 0 & 0\\
0 & -1 & 0\\
0 & 0 & 1
\end{pmatrix} \cdot \vec{p}=
\begin{pmatrix}
p_1\\
-p_2\\
p_3
\end{pmatrix}
\end{aligned}
 \stopformula

\stopsectionlevel

\startsectionlevel[title={Drehung um die
x_2-Achse},reference={drehung-um-die-x_2-achse}]

\startformula 
\begin{aligned}
\vec{\dot{p}}=
\begin{pmatrix}
\cos{\phi} & 0 & -\sin{\phi}\\
0 & 1 & 0\\
\sin{\phi} & 0 & \cos{\phi}
\end{pmatrix} \cdot \vec{p} =
\begin{pmatrix}
p_1\cos{\phi} & & -p_3\sin{\phi}\\
& p_2 & \\
p_1\sin{\phi} & & + p_3 \cos{\phi}
\end{pmatrix}
\end{aligned}
 \stopformula

\stopsectionlevel

\stopsectionlevel

\startsectionlevel[title={Aufgaben S. 202},reference={aufgaben-s.-202}]

\startsectionlevel[title={Matrix um zum Ursprung zu
Spiegeln},reference={matrix-um-zum-ursprung-zu-spiegeln}]

\startformula 
\begin{aligned}
\begin{pmatrix}
-1 & 0 & 0\\
0 & -1 & 0\\
0 & 0 & -1
\end{pmatrix}
\end{aligned}
 \stopformula

\stopsectionlevel

\stopsectionlevel

\startsectionlevel[title={Links:},reference={links}]

\startitemize[packed]
\item
  {[}{[}dreidimensionale_bildbearbeitung-pres{]}{]}
\stopitemize

\stopsectionlevel

\stoptext
