
\starttext

\setvariables[meta]
  [
  title={Binärsystem},
author={Niklas von Hirschfeld},
date={2024-08-12},
  ]

\MyTitlePage


% Start page numbering here
\setuppagenumber[number=1] % Reset and start numbering from 1
\setuppagenumbering[location={footer, middle}, alternative=singlesided, state=start]

\startsectionlevel[title={Aufbau},reference={aufbau}]

{\bf Bi}när kommt von \quotation{zwei zuständen}
{[}@Binary_advantages{]} . Diese können beliebig dargestellt werden,
üblicherweise werden sie aber mit \quotation{0} und \quotation{1}
dargestellt. Auf einer CPU werden sie durch physische gatter
dargestellt, welche entweder auf oder zu sind. Diese werden mit logik
gattern, wie AND und OR, erweitert um komplexere Rechnungen
durchzuführen.

\stopsectionlevel

\startsectionlevel[title={Unterschied zum
Dualsystem},reference={unterschied-zum-dualsystem}]

\stopsectionlevel

\startsectionlevel[title={Rechengesetze},reference={rechengesetze}]

Wie in den meisten Zahlensystemen gibt es Rechengesetze. Es wird jeweils
stellenweise gerechnet.

\startsectionlevel[title={Addition},reference={addition}]

Die Addition kann, ähnlich wie bei dem Dezimalsystem, schriftlich und
{\bf stellenweise} durchgeführt. Dafür gibt es vier Fälle, je nachdem,
welche zwei Ziffern addiert werden.

\startsectionlevel[title={Fall 1},reference={fall-1}]

\startformula 
0+0=0
 \stopformula

\stopsectionlevel

\startsectionlevel[title={Fall 1},reference={fall-1-1}]

\startformula 
0+1=1
 \stopformula

\stopsectionlevel

\startsectionlevel[title={Fall 1},reference={fall-1-2}]

\startformula 
1+0=1
 \stopformula

\stopsectionlevel

\startsectionlevel[title={Fall 1},reference={fall-1-3}]

\startformula 
1+1=0
 \stopformula

\startitemize[packed]
\item
  Übertrag von 1
\stopitemize

\stopsectionlevel

\stopsectionlevel

\startsectionlevel[title={Subtraktion},reference={subtraktion}]

\stopsectionlevel

\startsectionlevel[title={Multiplikation},reference={multiplikation}]

\stopsectionlevel

\startsectionlevel[title={Division},reference={division}]

\stopsectionlevel

\stopsectionlevel

\stoptext
