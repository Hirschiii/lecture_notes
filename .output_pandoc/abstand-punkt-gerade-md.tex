

\setvariables[meta]
  [
  title={Abstand eines Punktes von einer Gerade},
author={Niklas von Hirschfeld},
date={2024-08-13},
  ]

\starttext

\startfrontmatter
\component c_titlepage

% Table of Contents
\component c_content

\stopfrontmatter

\startbodymatter
\startsectionlevel[title={Funktionsweise},reference={funktionsweise}]

Wenn ein Punkt und eine Gerade gegeben ist, und der Abstand beider
gesucht, muss zunächst, wie auch bei dem Abstand eines Punktes von einer
Ebene, die kürzeste Distanz ermittelt werden. Bei einer Ebene konnte man
einfach einen Normalenvektor berechnen und damit eine Gerade aufstellen,
welche durch den Punkt geht. Bei einer Gerade ist es nicht gegeben, dass
es für jeden Normalenvektor zu der Gerade eine Gerade gibt, die auch
durch den Punkt geht.

Die kürzeste Distanz kann mithilfe einer Ebene gefunden werden. Wir
stellen also eine ebene auf, die folgende Bedingungen erfüllen muss:

\startitemize[packed]
\item
  Sie muss zur Gerade {\bf orthogonal} sein.
\item
  Sie muss den {\bf Punkt} beinhaltet.
\stopitemize

\startMPcode
  draw fullcircle scaled 3cm;
\stopMPcode

\stopsectionlevel

\stopbodymatter

\startbackmatter
\stopbackmatter

\stoptext
