\starttext
\chapter[title={2024-08-07 - Kegelspiel},reference={Kegelspiel}]

\startitemize[packed]
\item
  ☐ TODO: Buch kaufen Das Muschelessen Pipa Taschenbuchverlag: ISBN
  978-3492274005
\stopitemize

\startsectionlevel[title={Aufgabe:},reference={aufgabe}]

S. 263-264

\startsectionlevel[title={1b) sprachliche Darstellungsweise des
Textes},reference={b-sprachliche-darstellungsweise-des-textes}]

\startParagraph[reference=DAs ist ein test,title=DAs ist ein test]

\startitemize[packed]
\item
  Der vergleich mit der Kegelbahn
  \startitemize[packed]
  \item
    Sie nehmen {\em alle} rollen ein.
  \item
    Sie sind die ausführenden und Opfer zu gleich
  \item
    Die Soldaten werde Materialisiert / entmenschlicht
  \item
    Sie sind keine Menschen, sonder Ressourcen
  \stopitemize
\item
  \quotation{Geräumig} und \quotation{gemütlich. Wie ein Grab.} (Z. 09)
  \startitemize[packed]
  \item
    Gegensatzt um einstieg in die Absurdität zu bieten
  \stopitemize
\item
  \quotation{ein Gewehr. Das hatte iner erfunden, damit man damit auf
  Menschen schießt.} (Z. 11-12)
  \startitemize[packed]
  \item
    Trockene darstellung des Gewehr
  \stopitemize
\stopitemize

\stopParagraph

\stopsectionlevel

\startsectionlevel[title={2) Analysieren sie den Dialog der
Protagonisten},reference={analysieren-sie-den-dialog-der-protagonisten}]

\startitemize[packed]
\item
  Einer gibt die Schuld ab. \quotation{Aber man hat es befohlen} (Z. 49)
\item
  Der andere behart auf \quotation{wir haben es getan} (Z. 50)
\stopitemize

-> Gewissensfrage

\startplacetable[location=none]
\startxtable
\startxtablehead[head]
\startxrowgroup[lastrow]
\startxrow
\startxcell Soldate 1: {\bf Reflexion} \stopxcell
\startxcell Soldate 2: {\bf Ignoranz} \stopxcell
\stopxrow
\stopxrowgroup
\stopxtablehead
\startxtablebody[body]
\startxrowgroup[lastrow]
\startxrow
\startxcell Scham / ideologische Prägung \stopxcell
\startxcell Vergnügen \stopxcell
\stopxrow
\stopxrowgroup
\stopxtablebody
\startxtablefoot[foot]
\stopxtablefoot
\stopxtable
\stopplacetable

\stopsectionlevel

\startsectionlevel[title={3) Erläutern},reference={erläutern}]

\ldots{} Sie das Sprachild des \quotation{Kegelspiels} und seine
Funktion für

\stopsectionlevel

\stopsectionlevel

\stoptext
