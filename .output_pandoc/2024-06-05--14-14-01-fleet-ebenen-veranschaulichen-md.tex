
\starttext
\chapter[title={Ebenen veranschaulichen},reference={Ebenen
veranschaulichen}]

Spurpunkte: Schnittpunkte der Ebene mit den Koordinaten{\bf achsen}

Spurgeraden: Schnittgeraden einer Ebene mit den Koordianten{\bf ebenen}

\startsectionlevel[title={Drei
Schnittpunkte},reference={drei-schnittpunkte}]

\startitemize[packed]
\item
  Jeweils die anderen $x$-Werte ($x_1,x_2,x_3$) gleich null setzten
  \startitemize[packed]
  \item
    Für den $x_1$ Schnittpinkt $x_2=x_3=0$ einsetzen
  \stopitemize
\stopitemize

\stopsectionlevel

\startsectionlevel[title={Wenn alle drei Spurpunkte bekannt
sind},reference={wenn-alle-drei-spurpunkte-bekannt-sind}]

\startformula 
E: \quad \frac{x_1}{u}+\frac{x_2}{v}+\frac{x_3}{w}=1
 \stopformula

wobei die Spurpunkte wie folgt aussehen:

\startformula 
S_1=\begin{pmatrix}
u\\0\\0
\end{pmatrix}\quad 
S_2=\begin{pmatrix}
0\\v\\0
\end{pmatrix}\quad 
S_3=\begin{pmatrix}
0\\0\\w
\end{pmatrix}\quad 
 \stopformula

\stopsectionlevel

\stoptext
