
\setvariables
  [metadata]
  [
  title={Eigenes Verschlüsselungsverfahren},
author={Niklas von Hirschfeld},
institute={Gymnasium Lüneburger Heide},
  ]

\starttext

\startcolumnset[twocolumn]

\startblock[title=Grundidee Caeserverfahren]

Die Grundidee des {\bf Caeserverfahren} ist die {\bf feste} verschiebung
der Buchstaben im Alphabet. Das Vigenere-Verfahren versucht dies durch
eine ebenfalls {\bf feste} abfolge von {\em Schlüsseln} zu erweitern.

Um das Caeserverfahren zu verbessern kann man den Schlüssel dynamischer
gestalten. Dafür kann man zum Beispiel:

\startitemize[packed]
\item
  Das Schlüsselwort abwechselnd vorwärts und rückwärts anwenden.
\item
  Jeden {\em Schlüssel} mit dem Vorheriegen verrechnen (z.B. addieren).
\item
  den Schlüssel basierend auf dem Text {\em generieren}
\item
  Die Schlüssellänge dynamisch gestalten
\stopitemize

Man kann auch verschiedene Verfahren auf einander {\em schichten}. Zum
Beispiel in dem man:

\startitemize[packed]
\item
  mehrere Schlüssel nacheinander anwenden.
\item
  einmal vorwärst und einmal rückwärts verschlüsselt.
\item
  den selben Schlüssel mehrmals und jeweils versetzt anwenden.
\item
  oder alles zusammen.
\stopitemize

\stopblock

\startblock[title=Dynamisch Textbasiert]

Ich habe mich entschieden den Schlüssel dynamisch, basierend auf dem
Text, zu {\em generieren}. Dafür gibt es ebenfalls verschiedene Ansätze.
Man kann zum Beispiel die bis her verschlüsselten Buchstaben mit
verrechnen, oder alle noch übrigen Buchstaben.

Mein Ansatzt ist es, die Anzahl der noch im zu verschlüsselnden Text
auftauchenden Buchstaben mit zu verwenden und diese mit einem
Schlüsselwort zu verknüpfen. Dadurch kann sich der Schlüssel nach jedem
entschlüsselten Buchstaben verändern.

\column

\stopblock

\startblock[title=Funktionsweise]

Beim entschlüsseln mit dem Schlüsselwort \quotation{KEY} werden zunächst
alle K's gezählt. Die anzahl dieser wird mit 11 (Position von K im
alphabet) addiert. Der erste Buchstabe wird also um so viele Stellen
verschoben. Sollte sich dabei die Anzahl der K's im restlichen Text
verändern, verändert sich auch der Schlüssel. Abgesehen davon wird wie
bei dem Vigenere-Verfahren fortgefahren.

Das Verschlüsseln muss {\bf rückwärts} geschehen, da die bereits
verschlüsselten Buchstaben eine Rolle für den Schlüssel spielen. Man
fängt also mit dem letzten Buchstaben an verschlüsselt diesen. Sollte
der nun codierte Buchstabe in dem Schlüsselwort vorhanden sein, wird die
Verschiebung für diesen angepasst.

\stopblock

\startblock[title=Sicherheit]

Die größte Schwierigkeit beim lösen des Codes ist die Länge des
Schlüssels. Bei dem Vigenere-Verfahren kann diese durch sich
wiederholende Sequenzen im Text annäherungsweise ermittelt werden. Dies
ist hier nicht möglich. Auch hat eine {\bf Häufigkeitsanalyse} eine
geringe erfolgswahrscheinlichkeit da auch hier erst die Schlüssellänge
benötigt wird.

\stopblock

\startblock[title=Nachteile]

Unter bestimmten bedingungen kann es durchaus vorkommen, dass diese
Verfahren {\em identisch zum Vigenere-Verfahren} funktioniert. Zum
Beispiel, wenn in dem verschlüsselten Text kein einzieges mal ein
Buchstabe aus dem Schlüsselwort auftaucht.

Auch ist diese Methode bei {\bf kürzeren} Texten nicht sehr effektiv.

\stopblock

\stopcolumnset
\stoptext
