
\setvariables
  [metadata]
  [
  title={Caesercode},
author={Niklas von Hirschfeld},
institute={Gymnasium Lüneburger Heide},
  ]

\starttext

\startcolumnset[twocolumn]

\startblock[title=Einleitung]

Der {\bf Caesercode} (auch Caeser-Verschlüsselung oder -Verschiebung)
ist ein {\em symmetrisches}\footnote{es wird sowohl für die Ver- wie
  auch Entschlüsselung der selbe Schlüssel verwendet.}
Verschlüsselungsverfahren, welches nach Julius Caesar benannt ist.
Dieser habe es für die Kommunikation mit seinen militärischen
Verbündeten genutzt.

Bei der Verschlüsselung wird jeder Buchstaben durch einen anderen
Buchstaben mit einem festen Abstand im Alphabet ersetzt.

\startMPcode

u := 1cm;
shift_value := 3; % Number of positions to shift in the Caesar cipher

% Function to calculate the shifted character code
vardef caesar_encrypt(expr char, shift) =
  if (char >= "a") and (char <= "z"):
    ((char - "a" + shift) mod 26) + "a"
  elseif (char >= "A") and (char <= "Z"):
    ((char - "A" + shift) mod 26) + "A"
  else:
    char % Non-alphabetic characters remain unchanged
  fi
enddef;

% Example word to encrypt
string word;
word := "MetaFun";

% Draw original word
draw textext("Original:") shifted (0, 2.5u);
for i=0 upto length(word)-1:
  draw textext(word[i]) shifted (i*u, 2u);
endfor;

% Draw arrows indicating encryption
for i=0 upto length(word)-1:
  drawarrow ((i*u, 1.8u)--(i*u, 1.2u)) withpen pencircle scaled 0.5pt;
endfor;

% Draw encrypted word
draw textext("Encrypted:") shifted (0, u);
for i=0 upto length(word)-1:
  numeric encrypted_char;
  encrypted_char := caesar_encrypt(word[i], shift_value);
  draw textext(char encrypted_char) shifted (i*u, 0.5u);
endfor;

% Label indicating the shift value
draw textext("Shift = " & decimal shift_value) shifted ((length(word)*u + 0.5u), u);

\stopMPcode

\stopblock

\stopcolumnset
\stoptext
