
\setvariables
  [metadata]
  [
  title={Caeser-code},
author={Niklas von Hirschfeld},
institute={Gymnasium Lüneburger Heide},
  ]

\starttext

\startcolumnset[twocolumn]

\startblock[title=Einleitung]

Der {\bf Caesercode} (auch Caeser-Verschlüsselung oder -Verschiebung)
ist ein {\em symmetrisches}\footnote{es wird sowohl für die Ver- wie
  auch Entschlüsselung der selbe Schlüssel verwendet.}
Verschlüsselungsverfahren, welches nach Julius Caesar benannt ist.
Dieser habe es für die Kommunikation mit seinen militärischen
Verbündeten genutzt. Nachrichten mussten über lange Distanzen
transportiert werden. Dabei passierte es nicht selten, dass solche
Nachrichten abgefangen wurden. Damit dabei keine vertraulichen
Informationen an den {\em Gegner} gerieten, wurde die
Caeser-Verschlüsselung genutzt.

\stopblock

\startblock[title=Vorteile]

Damals war dieses Verschlüsselungsmethode gut und effektiv. Die größte
Vorteil war zu der darmaliegen Zeit die {\bf Unbekanntheit}. Wenn ein
solches Verfahren noch nicht bekannt oder Verbreitet ist, gibt es
weniger ansetze und intresse es zu knacken. Ein weiter Vorteil ist die
{\bf schnelle Ver- und Entschlüsselung}, wodurch die Infomation schnell
Menschenlesbar gemacht und genutzt werden können.

\stopblock

\startblock[title=Sicherheit]

Mittlerweile gilt dieses Verfahren als {\bf nicht Sicher}. Durch ihre
begrenzte Anzahl an Schlüsseln, ist ein lösen des Codes mithilfe der
{\em Bruteforcemethode}\footnote{Ausprobieren von allen möglichen
  Schlüsseln} durchaus realistisch.

Da dieses Verfahren schon alt und auch relative einfach ist, gibt es
mittlerweile viele gute und schnelle Wege, den Code zu lösen. Die
gängigsten sind eine {\bf Bruteforce-Attacke} oder eine {\bf
Häufigkeitsanalyse}.

\column

Beim {\bf Brutforce} werden einfach {\bf alle} möglichen Schlüssel
ausprobiert. Bei aktuellen und herkömmlichen Verschlüsselungsmethoden
dauert diese Attacke in der Theorie oft mehrere Jahrzehnte, auch mit den
aktuellsten Computern. Beim Caeser-Verfahren sollte es allerdings nicht
länger als Minuten oder sogar Sekunden dauern, da die Anzahl an
möglichen Schlüsseln bei 26 liegt. Zwar ist auch ein Schlüssel wie 27
{\em möglich} allerdings funktioniert dieser exakt wie der Schlüssel 1.

Bei der {\bf Häufigkeitsanalyse} geht es darum, die Anzahl der
auftauchenden Buchstaben zu analysieren. Diese Vergleicht man dann mit
der Häufigkeit des jeweiliegen Buchstaben in der ziel Sprache generell.
Im deutschen ist der am häufigsten auftauchnde Buchstabe das $e$. Wenn
jetzt ein Buchstabe am häufigsten auftaucht, ist es mit hoher
wahrscheinlichkeit das verschlüsselte {\em e}.

\stopblock

\startblock[title=Tools]

Den Prozess des Codieren können verschiedene Werkzeuge oder auch Scripte
vereinfach und verschnellern. Hier abgebildet ist Papierkonstrukt,
bestehend aus einer großen und einer kleineren Scheibe.

\framed[align=middle, width=\hsize, frame=off, height=fit]{
\startMPcode{doublefun}
u := .8cm; % Unit size
outer_radius := 3.5u;
inner_radius := 2.5u;

% Draw the outer circle
draw fullcircle scaled (2*outer_radius);

% Draw the inner circle
draw fullcircle scaled (2*inner_radius);

% Define the alphabets
string alphabet, small_alphabet;
alphabet := "ABCDEFGHIJKLMNOPQRSTUVWXYZ";
small_alphabet := "abcdefghijklmnopqrstuvwxyz";

% Draw the letters on the outer circle
for i=0 upto length(alphabet)-1:
    draw textext(substring(i, i+1) of alphabet) rotatedaround(origin, -i*13.846 - 90) shifted ((outer_radius - 0.3u)*dir(-i*13.846));
endfor;

%% Draw the letters on the inner circle
for i=0 upto length(small_alphabet)-1:
    draw textext(substring(i, i+1) of small_alphabet) rotatedaround(origin, -i*13.846 - 90) shifted ((inner_radius - 0.3u)*dir(-i*13.846));
endfor;
%
%% Draw the center point (pivot)
fill fullcircle scaled 0.1u shifted origin withcolor black;
%
%% Draw the arrow for rotation
path arrowhead;
arrowhead := (0, 0.5u) -- (0.4u, 0.7u) -- (0.4u, 0.3u) -- cycle;
\stopMPcode
}

Die Buchstaben sind nach dem Alphabet angeordnet und somit kann die
inner Scheibe weitergedreht werden, um eine neuen Schlüssel
darzustellen.

\stopblock

\stopcolumnset
\stoptext
