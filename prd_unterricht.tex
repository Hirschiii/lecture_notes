\startproduct prd_unterricht
\project project_school
\environment env_document


\startmode[how_to]
\setvariables[meta][title={HowTo}]
\setvariables[notes][path=how_to/unterricht]
\stopmode

\startmode[mathe]
\setvariables[meta][title={Mathematik}]
\setvariables[notes][path=mathe/unterricht]
\stopmode

\startmode[informatik]
\setvariables[meta][title={Informatik}]
\setvariables[notes][path=informatik/unterricht]
\stopmode

\startmode[physik]
\setvariables[meta][title={Physik}]
\setvariables[notes][path=physik/unterricht]
\stopmode

\startmode[englisch]
\setvariables[meta][title={Englisch}]
\setvariables[notes][path=englisch/unterricht]
\stopmode

\startmode[politik]
\setvariables[meta][title={Politik}]
\setvariables[notes][path=politik/unterricht]
\stopmode

\startmode[deutsch]
\setvariables[meta][title={Deutsch}]
\setvariables[notes][path=deutsch/unterricht]
\stopmode

\startmode[wun]
\setvariables[meta][title={Werte und Normen}]
\setvariables[notes][path=wun/unterricht]
\stopmode

\setvariables[meta]
  [
   subtitle={Unterricht - Abitur 2025},
   date={},
   author={Niklas von Hirschfeld},
  ]



\starttext

\startfrontmatter
\MyTitlePage
\completecontent
\stopfrontmatter

% Start page numbering here

\startbodymatter

\startluacode
	local path = tokens.getters.macro(tokens.getters.macro("??variables") .. "notes:path")
    include_markdown_files(path, "./.pandoc_templates/lecture-notes.context")
\stopluacode

\stopbodymatter



\stoptext
\stopproduct
